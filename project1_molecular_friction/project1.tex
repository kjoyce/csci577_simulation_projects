\documentclass[12pt]{amsart}
\usepackage[margin=1in]{geometry}
%\usepackage{tikz}
%\usetikzlibrary{calc}
\usepackage{amsmath, amsthm, amssymb, graphicx, setspace}
\usepackage{mymacros}
\usepackage{pysyntax}
\usepackage{verbatim}
\usepackage{hyperref}
\usepackage{graphicx}
\usepackage{caption}
\usepackage{subcaption}

\newcommand{\kry}{\mathcal K}
\newcommand{\figref}[1]{\figurename~\ref{#1}}

\title{A Microscopic Model of Friction}
\author{Kevin Joyce}
\date{25 March 2013} 

\begin{document}
\maketitle
\section{Introduction}

Usually, the first introduction to the sliding force of friction is modeled on
a macroscopic scale by what are known as Amontons' laws of friction - (1) The
force of friction is proportional to the the normal component of an applied
force, and (2) that this force is independent of the surface area of
contact\cite{ringlein}.  This model serves for a variety of surface materials,
yet may be counter-intuitive when surfaces with ``stickiness'' are considered,
e.g. the adhesive side of tape or the rubber sole of a tennis shoe.  Indeed,
these serve as apparent counter-examples to the general principle of Amontons'
law (2), and are usually explained by adding a force of adhesion to the model
\cite{ringlein} to balance the equation.  

When trying to develop these laws from first principles at a molecular level,
the intuition of ``stickiness'' is analogous to atomic forces between atoms, and the
notion of adhesion in order to preserve Amontons' law is logically circular. 
The more appropriate perspective is to adapt a more general model that
includes a ``sticky'' force and investigate why the area dependence dissipates
as the model scales to a macroscopic level.  This question is explored in
detail in \cite{ringlein} by constructing a computer simulation of simple
microscopic model for these ``sticky'' molecular forces.

In this short paper, we detail the results of replicating the model described in
\cite{ringlein} via the outline provided in \cite{gould}.  We describe the
details of this implementation in the programming language \texttt{Python} with
the scientific libraries provided by \texttt{SciPy}.  The
source code for this implementation with codes to animate the data and
visualize forces is freely available to download and modify at
\url{https://github.com/kjoyce/csci577_simulation_projects/tree/master/project1_molecular_friction}.
   
\section{Methods and Verification} 

Our implementation is built on top of the libraries \texttt{SciPy} \cite{scipy}
with the fundamental data-type describing the position, velocity, and force on
each particle being the so-called \texttt{NumPy} array.  

Integrations of position and velocity are accomplished through the second order Verlet algorithm:
\begin{python}
  a = f(x,v,t)
  dx = (v + .5*a*dt)*dt
  dv = .5*(f(xn,vntemp,t+dt) + a)*dt  
\end{python}
In the above codes, $f$ represents the right-hand-side of the equations governing acceleration.  This is given by the cumulative force on each particle divided by its respective mass (in our case all masses are assumed to be 1).

 Our implementation of forces follows the outline provided in
\cite{gould}, and is built upon a more general framework for modeling the
interaction of an $N$-particle system with interactions governed by the Lennard-Jones potential\cite{ringlein},
$$
  U(r) = 4\epsilon[(\sigma/r)^{12} - (\sigma/r)^6].
$$
In our simulation, we select parameter values of $\epsilon=1$ and $\sigma=1$.
All quantities are assumed to be appropriately non-dimensionalized as outlined
in \cite{ringlein}.  The cumulative force on each particly by the other $N-1$
particles is calculated by storing the pair-wise signed vector distances
between the $j$-th particle and the $i$-th particle in an $N\times N$ matrix (diagonal entries are ignored).  We can recover the force of
particle $j$ on $i$ by evaluating the gradient of $U$ on $r_{ij}$ of the signed
distance matrix.  By the principle of superposition, the total force on
particle $i$ is given by 
$$
  \sum_{j\not=i} \nabla U( \vect r_{ij} ) = \sum_{i\not=j} -\frac{24}{\vect r_{ij}} \left[2\|\vect r_{ij}\|^{-12} - \|\vect r_{ij}\|^{-6}\right]\vect r_{ij}.
$$ 
This computation is particularly suited to the \texttt{NumPy} array data-type.

The correctness of the Lennard-Jones force with the Verlet Algorithm was tested
on various symmetric configurations of particles (details available in the source codes).  We also implemented
visualizations of the implementations with the libraries provided in
\texttt{Matplotlib}. The results of these configurations can be viewed with the
Animation class also provided in the source codes.  

We model a one dimensional surface by a triangular lattice of particles on top of an immovable stationary line of particles.  The stationary particles are spaced at a distance of $a=2^{1/6}$ apart, and the particles in the triangular lattice are spaced $2a$ apart from each other (see \figref{lattice}).  The triangular lattice of particles is held together with stiff springs ($k=500$). The left-most particle is damped with a force equal to $\vect f_d = -10(v_x,v_y)$ to help stabilize the motion.  The right-most particle is pulled horizontally by a spring ($k=40$) moving at a rate of $.1t$.  I.e.
$$
  \vect f_p = 40(.1t - u(t),0),
$$
where $u$ is the the displacement of the right-most particle from the initial time.  A normal force is simulated by equally distributing various values of $W$ onto each particle of the lattice.  Each of the auxiliary forces were tested individually for correctness.  For example, the forces holding the lattice together were tested by turning the Lennard-Jones potential and other auxiliary forces (the drag force was tested by adding an initial velocity).
\begin{figure}
\includegraphics[width=.9\textwidth]{{lattice}.pdf}
\caption{The triangular lattice configuration modeling a one dimensional surface pulled horizontally on the right-most particle with a moving stiff spring ($k=40$) and stabilizing velocity-damping on the left-most particle by $-10(v_x,v_y)$.}
\label{lattice}
\end{figure}

The interaction off all of the forces was tested by animating the position of the particles of the lattice.  In addition, each simulation plots the force on the right-most particle with respect to time and the average velocity of the lattice with respect to time.  An example of one of these plots is in \figref{forcevel}.
\begin{figure}[t]
  \begin{subfigure}{0.49\linewidth}
    \includegraphics[width=\textwidth]{pullforce_sled13_load10_time8000.pdf}
  \end{subfigure}
  \begin{subfigure}{0.49\linewidth}
    \includegraphics[width=\textwidth]{avevelocity_sled13_load10_time8000.pdf}
  \end{subfigure}
  \caption{ The force of the moving spring on the right-most particle with respect to time and the average velocity of the lattice of a lattice configuration of 13 particles, loaded by $W=10$ over 8000 time steps ($dt=.01$). }
  \label{forcevel}
\end{figure}

\section{Analysis and Results}

\begin{figure}[h]
\begin{subfigure}{\textwidth}
\includegraphics[width=.16\textwidth]{pullforce_sled17_load-20_time8000.pdf}
\includegraphics[width=.16\textwidth]{pullforce_sled17_load-15_time8000.pdf}
\includegraphics[width=.16\textwidth]{pullforce_sled17_load-10_time8000.pdf}
\includegraphics[width=.16\textwidth]{pullforce_sled17_load-5_time8000.pdf}
\includegraphics[width=.16\textwidth]{pullforce_sled17_load0_time8000.pdf}
\includegraphics[width=.16\textwidth]{pullforce_sled17_load10_time8000.pdf}

\includegraphics[width=.16\textwidth]{pullforce_sled17_load15_time8000.pdf}
\includegraphics[width=.16\textwidth]{pullforce_sled17_load20_time8000.pdf}
\includegraphics[width=.16\textwidth]{pullforce_sled17_load25_time8000.pdf}
\includegraphics[width=.16\textwidth]{pullforce_sled17_load30_time8000.pdf}
\includegraphics[width=.16\textwidth]{pullforce_sled17_load35_time8000.pdf}
\includegraphics[width=.16\textwidth]{pullforce_sled17_load40_time8000.pdf}
\caption{ Triangular lattice with 17 particles } 
\end{subfigure}

\begin{subfigure}{\textwidth}
\includegraphics[width=.16\textwidth]{pullforce_sled13_load-20_time8000.pdf}
\includegraphics[width=.16\textwidth]{pullforce_sled13_load-15_time8000.pdf}
\includegraphics[width=.16\textwidth]{pullforce_sled13_load-10_time8000.pdf}
\includegraphics[width=.16\textwidth]{pullforce_sled13_load-5_time8000.pdf}
\includegraphics[width=.16\textwidth]{pullforce_sled13_load0_time8000.pdf}
\includegraphics[width=.16\textwidth]{pullforce_sled13_load10_time8000.pdf}

\includegraphics[width=.16\textwidth]{pullforce_sled13_load15_time8000.pdf}
\includegraphics[width=.16\textwidth]{pullforce_sled13_load20_time8000.pdf}
\includegraphics[width=.16\textwidth]{pullforce_sled13_load25_time8000.pdf}
\includegraphics[width=.16\textwidth]{pullforce_sled13_load30_time8000.pdf}
\includegraphics[width=.16\textwidth]{pullforce_sled13_load35_time8000.pdf}
\includegraphics[width=.16\textwidth]{pullforce_sled13_load40_time8000.pdf}
\caption{ Triangular lattice with 13 particles } 
\end{subfigure}


\begin{subfigure}{\textwidth}
\begin{center}
\includegraphics[width=.16\textwidth]{pullforce_sled9_load0_time8000.pdf}
\includegraphics[width=.16\textwidth]{pullforce_sled9_load10_time8000.pdf}
\includegraphics[width=.16\textwidth]{pullforce_sled9_load15_time8000.pdf}
\includegraphics[width=.16\textwidth]{pullforce_sled9_load20_time8000.pdf}

\includegraphics[width=.16\textwidth]{pullforce_sled9_load25_time8000.pdf}
\includegraphics[width=.16\textwidth]{pullforce_sled9_load30_time8000.pdf}
\includegraphics[width=.16\textwidth]{pullforce_sled9_load35_time8000.pdf}
\includegraphics[width=.16\textwidth]{pullforce_sled9_load40_time8000.pdf}
\caption{ Triangular lattice with 13 particles } 
\end{center}
\end{subfigure}

\caption{ These plots summarize the pulling force with respect to time. Note that the frequency of oscillations seems to be independent of the load and configuration.  The y-scales are not the same in each plot.  }
\label{allplots}
\end{figure}

\begin{figure}[H]
\includegraphics[width=.7\textwidth]{load_maxforce.pdf}
\caption{The maximum pull force versus the average load per particle. 
We note that the maximum statistic is quite sensitive to outlying data, which
may explain the seeming lack of fit of the data-point cooresponding to the load
of 15 on the 17-configuration. Future analysis might include estimates for each
peak in the simulation.  
}
\label{forceload}
\end{figure}

Simulations were preformed for the triangular lattice configurations with 9,
13, and 17 particles.  For each configuration we simulated an equally
distributed load, $W$, ranging from -20 to 40 by 5.  The force to overcome
friction was estimated by taking the maximum value over 8000 integration time
steps with $dt=.01$.  The inteval of $8000$ was choosen by initially simulating
the maximum load ($W=40$) at successive time lengths until a suitable number of
oscillations was observed in the force versus time plots.  

We remark that the effect of adding loads to the lattice can be interpreted as
subjecting the the particles to a graviational field that points downward.
Hence, negative loads merely represent a weaker gravitational field.  However,
this interpretation becomes invalid once the negative loading force
overcomes the Lennard-Jones force.  This situation occurs for the lattice of 9
particles, so negative loads were omitted in this situation.

In \figref{allplots}, notice that the general characteristic of force building
(nearly linearly) until the lattice becomes ``un-stuck'' and jerks forward can
be seen in each case.  Moreover, the frequency for which this patter repeats is
nearly the same throughout each plot.  This is to be expected, since the
pulling force is a spring whose force is proportional to the distance from the
pulling node.  The main feature of this figure is that it provides evidence
that the simulation ran correctly for each case.  

More telling is \figref{forceload}, in which we can see that the force required
to pull the lattice increases nearly linearly as load increases.  Note that the
lines appear to be parallel in the case of each configuration.  The results of
least square fitting are given in table \ref{lstsq}

\begin{table}[H]
\begin{tabular}{c| l c c }
lattice		& intercept & slope\\ \hline
9-configuration & \texttt{15.2311} & \texttt{0.3767}\\
13-configuration& \texttt{20.6507} & \texttt{0.3719}\\
17-configuration& \texttt{26.3260} & \texttt{0.3494}\\
\end{tabular}
\caption{Least-square fit to each configuration. }
\label{lstsq}
\end{table}


This provides evidence for two positive linear relationships - one for the average load versus
maximum pull force and the other for the surface area versus maximum pull force. The
linearity of the relationship between surface area and maximum pull force is inferred from the 
roughly linear relationship between the three intercepts (and the increase of 4 particles per simulation). Again we refer to \figref{forceload} to illustrate  this relationship.
 
\section{Conclusion}

Our simulation agrees with the results in \cite{ringlein}, in that we have observed a positive relationship between molecular surface area and the force of static friction (linear increase of intercepts in \figref{forceload}). This relationship appears to be linear based on the three linear increase of surface area in the three simulations.  This confirms the implied intuitin in the introduction that implies a model of static friction contrary to Amonton's Law (2).  We also observed a linear relationship between an applied normal force and the force of static friction (fit of points to line in \figref{forceload}). These results together suggest that a more appropriate model for the force of molecular static friction is
$$
  F_s = \mu_s W + c A,
$$ 
where $W$ is the applied normal force,  $A$ is the surface area of contact, and $\mu_s$ and $c$ are constants of proportionality.  In our simulation we predicted these vales to be $\hat \mu_s = 0.3767$ and $\hat c = 1.0$ by averaging predicted slopes and fitting a line through the intercepts of lines in \figref{forceload}.

The next question is why $c = 0$ on a macroscopic scale, reconciling with
Amontons' original model of friction.  This question is addressed in
\cite{ringlein}, and they attribute the lack of area dependence to the fact
that macroscopic surfaces are quite ``rough'' and there is relatively sparse
area of actual contact between atoms.  This reconciles with intuition regarding
adhesion, in that adhesion becomes important when materials easily deform as in tape
or rubber \cite{ringlein}.  

A future model for this situation might
be to model the contact surfaces in a probabilistic way so that local maxima
are randomly distributed on each contact surface.  Although our model codes
allows for three dimensional particle simulations, it does not scale well
beyond about 256 particles due to the distance calculation.  Future
optimization may make this simulation possible.



\begin{thebibliography}{[1]}
  \bibitem{gould} Gould, H. and Tobochnik J. and Christian, W. \emph{An Introduction to Computer Simulation Methods Second Edition Applications to Physical Systems},  Addison Wesley Publishing Group 2006, pg. 279-280.
  \bibitem{ringlein} Ringlein, J. and Robbins, M.O. \emph{Understanding and illustrating the atomic origins of friction}  American Journal of Physics, {\bf 72}, 884 (2004).
  \bibitem{scipy} Eric Jones and Travis Oliphant and Pearu Peterson and others, \emph{{SciPy}: Open source scientific tools for {Python}}, 2001--, \url{http://www.scipy.org/}.
  \bibitem{matplotlib} Hunter, J.D. \emph{Matplotlib: A 2D Graphics Environment}, Computing in Science \& Engineering, Vol. 9, No. 3. (2007), pp. 90-95.
\end{thebibliography}
\end{document}
